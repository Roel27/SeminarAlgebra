%
% wurzel.tex
%
% (c) 2021 Prof Dr Andreas Müller, OST Ostschweizer Fachhochschule
%
\documentclass[tikz]{standalone}
\usepackage{times}
\usepackage{amsmath}
\usepackage{txfonts}
\usepackage[utf8]{inputenc}
\usepackage{graphics}
\usetikzlibrary{arrows,intersections,math}
\usepackage{ifthen}
\begin{document}

\newboolean{showgrid}
\setboolean{showgrid}{true}
\setboolean{showgrid}{false}
\def\breite{7}
\def\hoehe{6}

\begin{tikzpicture}[>=latex,thick]

\clip (-6.3,-5.5) rectangle (6.3,5.15);

% Povray Bild
\node at (0,0) {\includegraphics[width=12.6cm]{wurzel.jpg}};

% Gitter
\ifthenelse{\boolean{showgrid}}{
\draw[step=0.1,line width=0.1pt] (-\breite,-\hoehe) grid (\breite, \hoehe);
\draw[step=0.5,line width=0.4pt] (-\breite,-\hoehe) grid (\breite, \hoehe);
\draw                            (-\breite,-\hoehe) grid (\breite, \hoehe);
\fill (0,0) circle[radius=0.05];
}{}

\node at (6.3,-2.2) [above left] {$\operatorname{Im} z$};
\node at (2.75,-5.05) [below] {$\operatorname{Re} z$};
\draw[->] (-4,0.2) -- (-4,-1.5);
\node at (-4,-0.65) [left] {$\pi$};
\node at (-5.55,-3.15) {$\mathbb{C}$};
\node at (-3.55,3.55) {$M$};

\end{tikzpicture}

\end{document}

