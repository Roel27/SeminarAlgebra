%
% 3-fortsetzung.tex -- Analytische Fortsetzung
%
% (c) 2025 Prof Dr Andreas Müller
%
\section{Analytische Fortsetzung
\label{buch:analytisch:section:fortsetzung}}
\kopfrechts{Analytische Fortsetzung}
Auf den ersten Blick scheint die Beschreibung einer holomorphen
Funktion durch eine Potenzreihe einen Verlust zu beinhalten.
Eine Potenzreihe ist ja nur innerhalb des Konvergenzkreises
konvergent, während eine holomorphe Funktion einen viel grösseren
Definitionsbereich haben kann.
Das Definitionsgebiet lässt sich aber immer mit Kreisen überdecken,
innerhalb derer sich die Funktion durch eine Potenzreihe schreiben
lässt.
In den Überschneidungsgebieten dieser Kreise stimmen die Funktionen
überein.
Der in diesem Abschnitt dargestellte Prozess der analytischen
Fortsetzung ermöglicht, eine holomorphe Funktion entlang einer
Kette von Kreisgebieten zu erweitern.
So lässt sich das grösstmögliche Definitionsgebiet
für eine holomorphe Funktion finden.
Es zeigt aber zum Beispiel bei der Logarithmus-Funktion auch, dass
die Menge der komplexen Zahlen zu klein sein kann und durch eine
Überlagerung ersetzt werden muss, um eine wohldefinierte komplex
differenzierbare Funktion zu konstruieren.

%
% Verschiedene Definitionsgebiete
%
\subsection{Verschiedene Definitionsgebiete}
Wir betrachten zwei Potenzreihen
\[
f_0(z)
=
\sum_{k=0}^\infty a^0_k(z-z_0)^k
\qquad\text{und}\qquad
f_1(z)
=
\sum_{k=0}^\infty a^1_k(z-z_1)^k
\]
mit Konvergenzradien $\varrho_0$ und $\varrho_1$.
Wir nehmen an, dass
\[
|z_1-z_0| < \varrho_0
\]
so dass der Punkt $z_1$ im Konvergenzkreis der Potenzreihe
$f_0(z)$ liegt.
Die Koeffizienten $a^1_k$ sind vollständig durch die Werte in
einer Umgebung von $z_1$ gegeben.
Um dies einzusehen, schreiben wir $z=(z-z_1) + (z_1-z_0)$ und drücken
\begin{align}
f_0(z)
&=
\sum_{n=0}^\infty a^0_n(z-z_0)^n
\notag
\intertext{durch $z-z_1$ aus und erhalten}
&=
\sum_{n=0}^\infty a^0_n((z-z_1)+(z_1-z_0))^n.
\notag
\intertext{Mit dem Binomialsatz kann dies in}
&=
\sum_{n=0}^\infty \sum_{k+l=n} a^0_n\binom{n}{k}(z_1-z_0)^{l} (z-z_1)^k 
\notag
\intertext{expandiert werden.
Da die innere Summe endlich ist, lässt sich die Summationsreihenfolge
ändern:
}
&=
\sum_{k=0}^\infty
\Biggl(
\sum_{n=k}^\infty
a^0_n\binom{n}{n-k}
(z_1-z_0)^{n-k}
\Biggr)
(z-z_1)^k
\notag
\intertext{Die Binomialkoeffizienten können als Bruch geschrieben werden,
was}
&=
\sum_{k=0}^\infty
\Biggl(
\sum_{n=k}^\infty
a^0_n
\frac{n\cdot(n-1)\cdot\ldots\cdot(n-k+1)}{k!}
(z_1-z_0)^{n-k}
\Biggr)
(z-z_1)^k
\notag
\intertext{ergibt.
Der Nenner $k!$ in den Termen der inneren Summe hängt nicht vom
Summationsindex ab und kann daher vor die Summe genommen werden:}
&=
\sum_{k=0}^\infty
\frac{1}{k!}
\Biggl(
\sum_{n=k}^\infty
a^0_n
\bigl(
n\cdot(n-1)\cdot\ldots\cdot (n-k+1)
\bigr)
(z_1-z_0)^{n-k}
\Biggr)
(z-z_1)^k.
\notag
\intertext{Das Produkt $n\cdot(n-1)\cdot\ldots\cdot(n-k+1)z^{n-k}$
ist die $k$-te Ableitung von $z^n$.
Da $k$ für alle Terme der inneren Summe gleich ist, ist die
innere Summe die $k$-te Ableitung einer einfacheren Summe,
nämlich}
&=
\sum_{k=0}^\infty
\frac{1}{k!}
\frac{d^k}{dz_1^k}
\Biggl(
\sum_{n=k}^\infty
a^0_n
(z_1-z_0)^n
\Biggr)
(z-z_1)^k.
\notag
\intertext{Die innere Summe ist nichts anderes als die Funktion
$f_0(z_1)$:}
&=
\sum_{k=0}^\infty
\frac{1}{k!}
\frac{d^k}{dz_1^k}
f_0(z_1)
(z-z_1)^k.
\label{buch:analytisch:fortsetzung:eqn:taylor01}
\end{align}
Die letzte Reihe ist die Taylor-Reihe der Funktion $f_0$ an der
Stelle $z_1$.

Wir nehmen jetzt an, dass in einer Umgebung vom Radius
$r_1-|z_0|$ die Funktionen $f_0(z)$ und $f_1(z)$ übereinstimmen.
Wir wissen bereits, dass die Koeffizienten $a^i_k$ durch die
Taylor-Reihe der Funktion $f_i(z)$ in den Punkten $z_i$ gegeben
ist.
Das Resultat
\eqref{buch:analytisch:fortsetzung:eqn:taylor01}
der obigen Rechnung bestätigt, durch einen Koeffizientenvergleich,
dass
\[
a^1_k
=
\frac{1}{k!}
\frac{d^kf_0(z_1)}{dz_1^k}.
\]

%
% Analytische Fortsetzung entlang einer Kurve
%
\subsection{Analytische Fortsetzung entlang einer Kurve}

%
% Überlagerungen der komplexen Ebene
%
\subsection{Überlagerungen der komplexen Ebene}

%
% Definitionsbereich für die Wurzelfunktionen
%
\subsubsection{Definitionsbereich für die Wurzelfunktionen}

%
% Definitionsbereich für die Logarithmusfunktion
%
\subsubsection{Definitionsbereich für die Logarithmusfunktion}

\subsubsection{Universelle Überlagerung von $\mathbb{C}\setminus\{0\}$}
