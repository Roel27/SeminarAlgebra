%
% 2-rational.tex -- Rationale Funktionen
%
% (c) 2025 Prof Dr Andreas Müller
%
\section{Rationale Funktionen
\label{buch:analytisch:section:rational}}
\kopfrechts{Rationale Funktionen}
Eine rationale Funktion ist ein Quotient der Form
\[
f(z)
=
\frac{p(z)}{q(z)},
\qquad
p(z),q(z)\in\mathbb{C}[z]
\]
von zwei Polynomen in $z$.
Polynome sind holomorphe Funktionen, die in ganz $\mathbb{C}$
definiert sind.
Nach der Quotientenregel ist auch der Quotient $f(z)$ eine holomorphe
Funktion überall dort, wo $q(z)$ nicht verschwindet.
Rationale Funktionen sind also auch analytisch.
In diesem Abschnitt untersuchen wir, wie sich für eine solche Funktion
eine Potenzreihenentwicklung finden lässt.

%
% Faktorisierung
%
\subsection{Faktorisierung}
Die Tatsache, dass sich Polynome über $\mathbb{C}$ faktorisieren
lassen, lässt sich dazu verwenden, eine Potenzreihe für eine
rationale Funktion zu konstruieren, ohne dass die Ableitungen
der Funktion explizit berechnet werden müssen.

%
% Entwicklung
%
\subsubsection{Entwicklung um $0$}
Das Zählerpolynom
\[
p(z)
=
p_0
+
p_1z
+
p_2z^2
+ \dots +
p_{n-1}z^{n-1}
+
p_nz^n
\]
ist bereits eine Potenzreihe um den Punkt $0$, die nach endlich
vielen Termen abbricht, deren Konvergenzradius unendlich ist.

Der Fundamentalsatz der Algebra besagt, dass sich das
Nennerpolynom in Linearfaktoren 
\begin{align*}
q(z) &= b(z - q_1)(z - q_2)\dots(z - q_m)
\end{align*}
zerlegen lässt. 
Die Zahlen $q_i\in\mathbb{C}$ sind die Nullstellen von $q(z)$.
Jeder einzelne Faktor der Form $z-a$ des Nennerpolynoms gibt
Anlass zu einer geometrischen Reihe der Form
\[
\frac{1}{z-a}
=
-a\cdot \frac{1}{\displaystyle 1-\frac{z\mathstrut}{a\mathstrut}}
=
-a
\sum_{k=0}^\infty
\frac{z^k}{a^k},
\]
die konvergiert, solange $|z/a|<1$ oder $|z| < |a|$ ist.
Dies ist eine Potenzreihenentwicklung im Punkt $0$.

Das Produkt von Potenzreihen ist wieder eine Potenzreihe.
Die Funktion $f(z)$ ist daher das Produkt
\begin{align*}
f(z)
&=
p(z)
\cdot
\frac{1}{b}
\prod_{k=1}^\infty
\frac{1}{z-q_k}
\\
&=
\frac{p(z)}{b}
\prod_{k=1}^\infty
\Biggl(
-q_k
\cdot
\frac{1}{\displaystyle 1-z/q_k}
\Biggr)
\\
&=
\frac{p(z)}{b}
(-1)^m
q_1\cdots q_m
\prod_{k=1}^\infty
\Biggr(
\sum_{i=0}^\infty
\frac{z^i}{q_k^i}
\Biggr).
\end{align*}
Es hat das Minimum
\[
\varrho 
=
\min
\{|q_i|\mid 1\le i\le m\}
\]
als Konvergenzradius.

%
% Entwicklung um einen beliebigen Punkt z_0
%
\subsubsection{Entwicklung um einen beliebigen Punkt $z_0$}
Das Polynom $p(z)$ kann auch als Potenzreihe um den Punkt $z_0$
geschrieben werden, wie dies in Abschnitt
\ref{buch:analytisch:potenzreihen:subsection:polynome}
gezeigt wurde.

