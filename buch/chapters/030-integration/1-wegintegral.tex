%
% 1-wegintegral.tex -- 1. Wegintegral
%
% (c) 2025 Prof Dr Andreas Müller
%
\section{Wegintegrale in $\mathbb{C}$
\label{buch:integration:section:wegintegral}}
\kopfrechts{Wegintegrale in $\mathbb{C}$}
Das Riemann-Integral einer Funktion ist definiert für Invervalle.
Mit Hilfe eines Weges im Definitionsgebiet einer komplexen Funktion
können wir die Funktion in eine komplexwertige Funktion auf einem
Intervall, nämlich dem Definitionsbereich des Weges verwandeln.
Da es zwischen zwei Punkten von $U$ beliebig viele verschiedene
Wege gibt, müssen die Bedingungen identifiziert werden, unter denen
das Integral nicht von der Wahl des Weges abhängt.

%
% Wege in $\mathbb{C}$
%
\subsection{Wege in $\mathbb{R}^n$ und $\mathbb{C}$}
Ein \emph{Weg} in einer offenen Menge $U\subset \mathbb{R}^n$ ist eine
\index{Weg}
stetige Abbildung
\[
\gamma : I \to U : t \mapsto \gamma(t)
\]
für ein Intervall $I=[a,b]$.
Der Weg heisst \emph{geschlossen}, wenn Anfangs- und Endpunkt übereinstimmen,
\index{Weg!geschlossen}!
\index{geschlossener Weg}!
wenn also $\gamma(a)=\gamma(b)$ gilt.
Ein Weg heisst \emph{differenzierbarer Weg}, wenn die Funktion $\gamma$
\index{Weg!differenzierbar}
differenzierbar ist.
Die Ableitung
\[
\dot{\gamma}
\colon
I
\to
\mathbb{R}^n
:
t
\mapsto
\dot{\gamma}(t)
=
\frac{d\gamma}{dt}(t)
\]
heisst auch der Tangentialvektor des Weges im PUnkt $\gamma(t)$.

Da $\mathbb{C}$ als zweidimensionaler reeller Vektorraum betrachtet
werden kann, ist ein Weg in $U\subset\mathbb{C}$ eine Abbildung
von einem Intervall $I$ nach $U$:
\[
\gamma
\colon
I
\to
U
:
t
\mapsto
\gamma(t).
\]
Die Ableitung nach $t$ ist ebenfalls eine komplexe Zahl, die Funktion
\[
\dot{\gamma}
\colon
I
\to
\mathbb{C}
:
t
\mapsto
\dot{\gamma}(t)
=
\frac{d\gamma}{dt}(t)
\]
heisst wieder Tangentialvektor, er ist jetzt aber auch eine komplexe 
Zahl.

Ein Weg in $\mathbb{R}^2$ entsteht dadurch, dass man den Vektor aus
Real- und Imaginärteil von $\gamma(t)$ bildet:
\[
\gamma_{\mathbb{R}}
\colon
I
\to
\mathbb{R}^2
:
t
\mapsto
\begin{pmatrix}
\operatorname{Re}\gamma(t)\\
\operatorname{Im}\gamma(t)
\end{pmatrix}.
\]
Der Tangentialvektor besteht aus den Ableitungen von Real-
und Imaginärteil
\[
\dot{\gamma}
\colon
I
\to
\mathbb{C}
:
t
\mapsto
\frac{d}{dt}\bigl(\operatorname{Re}\gamma(t)\bigr)
+
i
\frac{d}{dt}\bigl(\operatorname{Im}\gamma(t)\bigr)
\]
nach $t$.

\input{chapters/030-integration/fig/fig-wege.tex}%

\begin{beispiel}
\label{buch:integration:wege:bsp:strecke}
Sei $U=\mathbb{C}$ und seien zwei Punkte $z_0,z_1\in \mathbb{C}$ gegeben.
Dann ist
\[
\gamma
\colon
[0,1]
\to
\mathbb{C}
:
(1-t)z_0 + tz_1
\]
ein differenzierbarer Weg, der $\gamma(0)=z_0$ mit $\gamma(1)=z_1$
verbindet.
Die Ableitung von $\gamma$ ist
\[
\dot{\gamma}(t)
=
z_1-z_0.
\]
Insbesondere ist die Ableitung konstant.

Weitere Wege mit den gleichen Endpunkten können konstruiert werden,
indem ein Weg addiert wird, der von $0$ nach $0$ führt.
Zum Beispiel ist für jedes $n\in\mathbb{N}$ 
\[
\delta(t)
=
r_0
(e^{2\pi int} - 1)
\]
ein kreisförmiger Weg mit Radius $r_0$, der $n$ Mal durchlaufen wird.
Fügt man diesen Weg und zustzlich den Summanden $i\sin \pi t$ zum Weg
$\gamma(t)$ hinzu, erhält man den Weg
\[
\gamma_1(t)
=
\gamma(t)
+
r_0
(e^{2\pi int} - 1)
+
i\sin \pi t,
\]
der in Abbildung~\ref{buch:integration:wege:fig:wege}~a)
{\color{darkgreen}grün} eingezeichnet ist.
Der zugehörige Geschwindigkeitsvektor 
\[
\dot{\gamma}_1(t)
=
(z_1-z_0)
+
2\pi i
n
r_0
e^{2\pi int}
+
i\pi\cos \pi t.
\qedhere
\]
\end{beispiel}

\begin{beispiel}
\label{buch:integration:wege:bsp:kreis}
Sei $U=\mathbb{C}$ und $r\in\mathbb{R}$.
Der Weg
\[
\gamma
\colon
[0,2\pi]
\to
U
:
t
\mapsto
r(\cos t + i\sin t)
\]
ist ein differenzierbarer Weg mit $\gamma(0) = \gamma(2\pi) = 1$.
Die Ableitung ist
\[
\dot{\gamma}(t)
=
r(-\sin t + i\cos t).
\]
In \eqref{XXX} 
wurde gezeigt, dass $\gamma(t) = re^{it}$ ist mit der Ableitung
$\dot{\gamma}(t)=rie^{it}$.
\end{beispiel}


%
% Integral entlang eines Weges
%
\subsection{Integral entlang eines Weges}
Für eine stetige komplexe Funktion $f(z)$ und einen Weg $\gamma(t)$
ist $f(\gamma(t))$ eine stetige Funktion der reellen Variablen $t$.
Nach allgemeinen Sätzen über das Riemann-Integral können wir also 
von Real- und Imaginärteil das Integral über das Definitionsgebiet
berechnen.
Die Definition eines Wegintegrals sollte aber möglichst nicht von
der Parametrisierung des Weges abhängen.
Das Integral der konstanten Funktion $1$ ist aber immer
\[
\int_a^b f(\gamma(t)) \,dt
=
\int_a^b 1\,dt
=
b-a,
\]
hängt also sehr wohl von der Parametrisierung ab.
Daher werden wir zunächst eine von der Parametrisierung unabhängige
Form des Wegintegrals definieren müssen.
Anschliessend werden wir zeigen, wie dieses Wegintegral mit der
Stammfunktion zusammenhängt.

%
% Definition des Wegintegrals
%
\subsubsection{Definition des Wegintegrals}
Das einführende Beispiel hat gezeigt, dass das Integral
von $f(\gamma(t))$ von der Parametrisierung abhängt.
Bewegt sich der Punkt $\gamma(t)$ langsamer durch das Gebiet $U$,
wird ein längeres Parameterintervall benötigt, um die Strecke
zwischen zwei Punkten zu durchlaufen.
Das gesuchte Wegintegral benötigt also auch noch einen
von der Geschwindigkeit $\dot{\gamma}(t)$ abhängigen Faktor,
um diesen Unterschied auszugleichen.

\begin{definition}[Wegintegral in $\mathbb{C}$]
Sei $f\colon U\to\mathbb{C}$ eine stetige komplexe Funktion und
$\gamma:I\to U$ ein differenzierbarer Weg $U$ mit $I=[a,b]$.
Dann ist das \emph{Integral von $f$ entlang des Weges $\gamma$}
\index{Integral entlang eines Weges}%
gegeben durch
\begin{equation}
\int_\gamma f(z)\,dz
=
\int_I f(\gamma(t)) \dot{\gamma}(t)\,dt.
\end{equation}
Die gleiche Definition kann verwendet werden, wenn der Weg
$\gamma$ in endlich vielen Punkten nicht differenzierbar ist.
Wenn $\gamma$ ein geschlossener Weg ist, wird das Integral auch
\[
\int_\gamma f(z)\,dz
=
\oint_\gamma f(z)\,dz
\]
geschrieben.
\end{definition}

Wir müssen überprüfen, dass diese Definition tatsächlich nicht
von der Parametrisierung abhängt.
Dazu sei $t(\tau)$ eine invertierbare Funktion, die auf einem Intervall
$I_1=[c,d]$ definiert ist.
Nach der Formel für den Koordinatenwechsel in einem Integral
folgt dann
\begin{align*}
\int_c^d f(\gamma(t(\tau))) \frac{d\gamma(t(\tau))}{d\tau} \,d\tau
&=
\int_c^d f(\gamma(t(\tau))) \dot{\gamma}(t(\tau)) \frac{dt(\tau)}{dt}\,d\tau
\\
&=
\int_{\tau(c)}^{\tau(d)} f(\gamma(t)) \dot{\gamma}(t) \,dt
\\
&=
\int_a^b f(\gamma(t)) \dot{\gamma}(t) \,dt.
\end{align*}
Die Umparametrisierung $\tau(t)$ ändert also den Wert des Integrals
nicht.

\begin{beispiel}
\label{buch:integration:wege:bsp:kreisintegral}
Wir berechnen das Wegintegral für den geschlossenen Weg von
Beispiel~\ref{buch:integration:wege:bsp:kreis}
für die Funktion $f(z) = z^n$.
Einsetzen der Parametrisierung ergibt
\begin{align*}
\oint_\gamma f(z)\,dz
&=
\int_0^{2\pi}
\bigl( e^{it} \bigr)^{n}\,d ie^{it}
\,dt
=
i
\int_0^{2\pi}
e^{i(n+1)t}
\,dt
=
i
\int_0^{2\pi} \cos (n+1)t + i\sin(n+1)t\,dt.
\end{align*}
Falls $n+1\ne 0$ ist, stehen auf der rechten Seite Integrale der
Sinus- und Kosinusfunktion über ein ganze Anzahl von Perioden, diese
Integrale verschwinden.
Für $n=-1$ jedoch verschwindet der Sinus-Term und es bleibt
\[
\oint_\gamma \frac{dz}z
=
i
\int_0^{2\pi}
1\,dt
=
2\pi i.
\]
Zusammengefasst erhalten wir für ganzzahliges $n$
\[
\oint_\gamma z^n\,dtz
=
\begin{cases}
0&\qquad\text{falls $n\ne -1$}\\
2\pi i&\qquad\text{falls $n=-1$}
\end{cases}
\qedhere
\]
\end{beispiel}

Wir schreiben $\gamma(t) = x(t) + iy(t)$ und die Funktion $f=u+iv$ 
als Summe von Real- und Imaginärteil.
Dann wird das Wegintegral zu
\begin{align*}
\int_\gamma f(z)\,dz
&=
\int_a^b \bigl(u(x(t), y(t)) + i v(x(t), y(t))\bigr)\,
(\dot{x}(t)+ i\dot{y}(t))\,dt
\\
&=
\int_a^b u(x(t),y(t)) \dot{x}(t) -  v(x(t),y(t))\dot{y}(t)\,dt
\\
&\qquad
+
i
\int_a^b u(x(t),y(t))\dot{y}(t) + v(x(t),y(t))\dot{x}(t)\,dt,
\end{align*}
und kann dank dieser Formeln allein mit reellen Integralen berechnet
werden.

%
% Wegintegrale über achsenparallele Wege
%
\subsubsection{Wegintegrale über achsenparallele Wege}
\input{chapters/030-integration/fig/fig-rechteck.tex}%
Wir betrachten eine stetige komplex Funktion $f\colon U\to\mathbb{C}$
und zwei Punkte $a,b\in U$.
Der Einfachheit halber nehmen wir an, dass das Rechteck $R$ mit Ecken $a$
und $b$ in $U$ enthalten ist.
Als Weg zwischen $a$ und $b$ können Teile des Randes des Rechtecks $R$
verwendet werden.

Der Weg $\gamma_1$ in Abbildung~\ref{buch:integration:wegintegral:fig:rechteck}
verläuft zunächst parallel zur reellen Achse, bevor er parallel zur
imaginären Achse in den Punkt $b$ führt.
Der horizontale Teil des Weges kann mit
\begin{align*}
\gamma_{1,h}
&\colon
[\operatorname{Re}a,\operatorname{Re}b]
\to
\mathbb{C}
:
t
\mapsto
t + i\operatorname{Im} a
&&\Rightarrow&
\dot{\gamma}_{1,h} &= 1
\intertext{parametrisiert werden, während der vertikale Teil hat
die Parametrisierung}
\gamma_{1,v}
&\colon
[\operatorname{Im}a,\operatorname{Im}b]
\to
\mathbb{C}
:
s
\mapsto
\operatorname{Re}b + is
&&\Rightarrow&
\dot{\gamma}_{1,h} &= i.
\end{align*}
Das Wegintegral der Funktion über den Weg $\gamma_1$ kann daher als
Summe zweier Integrale
\begin{align*}
F_1(b)
&=
\int_{\gamma_1} f(z)\,dz
\\
&=
\int_{\gamma_{1,h}} f(z)\,dz
+
\int_{\gamma_{1,v}} f(z)\,dz
\\
&=
\int_{\operatorname{Re}a}^{\operatorname{Re}b}
f(t+i\operatorname{Im}a)
\,dt
+
\int_{\operatorname{Im}a}^{\operatorname{Im}b}
f(\operatorname{Re}b+is)
\cdot
i
\,ds
\end{align*}
Die Ableitung von $F_1(b+iy)$ nach $y$ kann damit ebenfalls berechnet
werden:
\begin{align*}
\frac{1}{i}
\frac{\partial}{\partial y} F_1(b+iy)
&=
\frac{1}{i}
\frac{\partial}{\partial y}
\int_{\operatorname{Im}a}^{\operatorname{Im}b+y}
f(\operatorname{Re}b+is)\cdot i
\,ds
=
\frac{\partial}{\partial y}
\int_{\operatorname{Im}a}^{\operatorname{Im}b+y}
f(\operatorname{Re}b+is)
\,ds
\\
&=
f(\operatorname{Re}+i(\operatorname{Im}b+y))
=
f(b)
\end{align*}
Da der Weg $\gamma_1$ im Punkt $b$ vertikal verläuft, kann man ihn
zum Zweck der Ableitung noch etwas über $\operatorname{Im}b$ hinaus
verlängern.
Er definiert so eine Funktion $F_1(b)$, deren
Ableitung nach $i\operatorname{Im}b$ mit $f(b)$ übereinstimmt.

Andererseits erlaubt der Weg $\gamma_2$ eine Funktion
\begin{align*}
F_2(b)
&=
\int_{\gamma_2} f(z) \,dz
\\
&=
\int_{\gamma_{2,v}}
f(z)\,dz
+
\int_{\gamma_{2,h}}
f(z)\,dz
\\
&=
\int_{\operatorname{Im}a}^{\operatorname{Im}b}
f(\operatorname{Re}a + is) \cdot i \,ds
+
\int_{\operatorname{Re}a}^{\operatorname{Re}b}
f(t + i\operatorname{Im}b)
\,dt
\end{align*}
Durch horizontale Verlängerung des Weges $\gamma_2$ über den Punkt $b$
hinaus kann man jetzt auch die Ableitung von $F_2(b+x)$ berechnen:
\begin{align*}
\frac{\partial}{\partial x}F_2(b+x)
&=
\frac{\partial}{\partial x}
\int_{\operatorname{Re}a}^{\operatorname{Re}b+x}
f(t + i\operatorname{Im}b)
\,dt
\\
&=
f(\operatorname{Re}b+x+i\operatorname{Im}b)
=
f(b+x).
\end{align*}
Der Weg $\gamma_2$ definiert also eine Funktion $F_2(b)$, deren
Ableitung nach $\operatorname{Re}b$ mit $f(b)$ überinstimmt.

Wenn wir zusätzlich zeigen können, dass die beiden Funktionen
$F_1(b)$ und $F_2(b)$ übereinstimmen, dann ist dadurch eine
Stammfunktion von $f$ definiert.

%
% Integral einer holomorphen Funktion über ein Rechteck
%
\subsubsection{Integral einer holomorphen Funktion über ein Rechteck}
In diesem Abschnitt nehmen wir zusätzlich an, dass Real- und Imaginärteil
als Funktionen von zwei reellen Variablen differenzierbar sind.
Wir möchten die Voraussetzungen finden, unter denen es nicht darauf
ankommt, welchen der zwei betrachteten Wege wir nehmen.
Wir möchten also untersuchen, wann die Differenz $F_1(b) = F_2(b)$
verschwindet.

Wir berechnen also die Differenz
\begin{align*}
F_1(b) - F_2(b)
&=
\int_{\operatorname{Re}a}^{\operatorname{Re}b}
f(t+i\operatorname{Im}a)
\,dt
+
\int_{\operatorname{Im}a}^{\operatorname{Im}b}
f(\operatorname{Re}b+is)
\cdot
i
\,ds
\\
&\qquad
-
\int_{\operatorname{Im}a}^{\operatorname{Im}b}
f(\operatorname{Re}a + is) \cdot i
\,ds
-
\int_{\operatorname{Re}a}^{\operatorname{Re}b}
f(t + i\operatorname{Im}b)
\,dt
\\
&=
-
\int_{\operatorname{Re}a}^{\operatorname{Re}b}
f(t + i\operatorname{Im}b)
-
f(t + i\operatorname{Im}a)
\,dt
\\
&\qquad
+
i
\int_{\operatorname{Im}a}^{\operatorname{Im}b}
f(\operatorname{Re}b+is)
-
f(\operatorname{Re}a+is)
\,ds
\end{align*}
Die Differenzen im Integranden können als Integral einer Ableitung
geschrieben werden:
\begin{align*}
f(\operatorname{Re}b+is)
-
f(\operatorname{Re}a+is)
&=
\int_{\operatorname{Re}a}^{\operatorname{Re}b}
\frac{\partial f}{\partial x}(x+is)
\,dx
\\
f(t + i\operatorname{Im}b)
-
f(t + i\operatorname{Im}a)
&=
\int_{\operatorname{Im}a}^{\operatorname{Im}b}
\frac{\partial f}{\partial y}(t+iy)
\,dy.
\end{align*}
Damit wird die Diffeernz
\begin{align*}
F_1(b) - F_2(b)
&=
-
\int_{\operatorname{Re}a}^{\operatorname{Re}b}
\int_{\operatorname{Im}a}^{\operatorname{Im}b}
\frac{\partial f}{\partial y}(x+iy)
\,dy
\,dx
+
i
\int_{\operatorname{Im}a}^{\operatorname{Im}b}
\int_{\operatorname{Re}a}^{\operatorname{Re}b}
\frac{\partial f}{\partial x}(x+iy)
\,dx
\,dy
\intertext{Die beiden Integrationen des doppelten Integrals können 
vertauscht werden, die Differenz ist also auch}
&=
\int_{\operatorname{Re}a}^{\operatorname{Re}b}
\int_{\operatorname{Im}a}^{\operatorname{Im}b}
-\frac{\partial f}{\partial y}(x+iy)
+
i
\frac{\partial f}{\partial x}(x+iy)
\,dy
\,dx
\\
&=
-2i
%\iint_{R}
\int\hspace*{-2mm}\int_{R}
\frac12
\biggl(
\frac{\partial f}{\partial x}
+
i
\frac{\partial f}{\partial y}
\biggr)
\,dx\,dy
=
-2i
\int\hspace*{-2mm}\int_R
%\iint_R 
\frac{\partial f}{\partial\bar{z}}
\,dx\,dy
\end{align*}
Die Differenz verschwindet also genau dann, wenn die Ableitung
$\partial f/\partial\bar{z}$ verschwindet, also genau für holomorphe
Funktionen.

\begin{satz}
\label{buch:integration:wegintegral:satz:rechteckintegrale}
Sei $f\colon U\to\mathbb{C}$ 
eine Funktion, deren Real- und Imaginärteile als Funktionen von zwei
reellen Variablen differenzierbar sind. 
Dann sind die Integrale über die Wege $\gamma_1$ und $\gamma_2$ für
beliebige Rechtecke mit den Ecken $a,b\in\mathbb{C}$, die in $U$
enthalten sind, genau dann gleich, wenn die Funktion $f$ holomorph
ist.
\end{satz}

%
% Der Satz von Green
%
\subsubsection{Der Satz von Green}
Die Aussage von
Satz~\ref{buch:integration:wegintegral:satz:rechteckintegrale}
ist ein Spezialfall des sehr viel allgemeineren Satzes von Green.

\begin{satz}
\label{buch:integration:wegintegral:satz:green}
Ist $f\colon U\to \mathbb{C}$ eine komplexe Funktion, deren Real-
und Imaginärteile als Funktionen zweier reeller Variablen differenzierbar
sind und $\gamma$ ein geschlossener Weg in $U$, der das Teilgebiet
$\Omega\subset U$ berandet.
Dann gilt
\[
\oint_\gamma f(z)\,dz
=
-2i
\int_\Omega \frac{\partial f}{\partial\bar{z}}\,dx\,dy.
\]
\end{satz}

\begin{proof}
Der Beweis verwendet genau die gleichen Methoden, die auch schon
im Beweis von
Satz~\ref{buch:integration:wegintegral:satz:rechteckintegrale}
erfolgreich waren.

TODO

\end{proof}

\begin{korollar}[Integralsatz von Cauchy]
\label{buch:integration:wegintegral:korollar:geschlossen}
Ist $f\colon U\to\mathbb{C}$ eine holomorphe Funktion und
$\gamma$ ein geschlossener Weg, der ein Teilgebiet $\Omega\subset U$
berandet, dann ist
\[
\oint_\gamma f(z)\,dz
=
0.
\]
\end{korollar}

\begin{korollar}
\label{buch:integration:wegintegral:korollar:geschlossen}
Sei $f\colon U\to\mathbb{C}$ eine holomorphe Funktion und $a,b\in U$
zwei Punkte, sodass auch das Rechteck mit Ecken $a$ und $b$ in $U$
enthalten ist.
Ausserdem sei $\gamma$ ein Weg von $a$ nach $b$, der innerhalb des
Rechtecks verläuft.
Dann ist
\[
F(b)
=
\int_\gamma f(z)\,dz
=
\int_{\gamma_1} f(z)\,dz
=
\int_{\gamma_2} f(z)\,dz.
\]
\end{korollar}

\begin{proof}
Verbindet man den Weg $\gamma_1$ mit dem in umgekehrten Richtung
durchlaufenen Weg $\gamma$, entsteht ein geschlossener Weg,
den wir mit $\gamma_1-\gamma$ beizeichnen.
Nach Korollar~\ref{buch:integration:wegintegral:korollar:geschlossen}
muss das Integral über den geschlossenen Weg verschwinden, also ist
\[
0
=
\int_{\gamma_1-\gamma}
f(z)\,dz
=
\int_{\gamma_1} f(z)\,dz
-
\int_\gamma f(z)\,dz
\quad\Rightarrow\quad
\int_\gamma f(z)\,dz
=
\int_{\gamma_1} f(z)\,dz.
\qedhere
\]
\end{proof}

%
% Wegintegral und Stammfunktion
%
\subsubsection{Wegintegral und Stammfunktion}
Der Satz~\ref{buch:integration:wegintegral:satz:green}
und insbesondere das
Korollar~\ref{buch:integration:wegintegral:korollar:geschlossen}
zeigen, dass es nicht darauf ankommt, welchen Weg man zwischen
zwei Punkten $a$ und $b$ wählt, solange zwei solche Wege ein
Teilgebiet des Definitionsgebietes beranden.
In diesem Fall lässt sich also durch Wegintegration immer eine
Stammfunktion finden.
Dazu muss es aber zwischen zwei beliebigen Punkten von $U$ einen
Weg geben.

\begin{definition}[wegzusammenhängend]
Ein Gebiet $U\subset\Omega$ heisst \emph{wegzusammenhängend}, wenn
\index{wegzusammenhangend@wegzusammenhängend}%
es zu je zwei Punkten $a,b\in U$ einen Weg gibt, der $a$ und $b$
verbindet.
\end{definition}

\begin{satz}[Stammfunktion]
Sei $U\subset\mathbb{C}$ ein wegzusammenhängendes Gebiet und $a\in U$
ein Punkt.
Aussserdem habe $U$ die Eigenschaft, dass für jeden Punkt $b\in U$
zwei verschiedene Wege, die $a$ und $b$ verbinden, ein offenes Teilgebiet
von $U$ beranden.
Dann hat eine holomorphe Funktion $f\colon U\to\mathbb{C}$ eine Stammfunktion
\[
F(b)
=
\int_\gamma f(z)\,dz,
\]
wobei $\gamma$ ein beliebiger Weg ist, der $a$ mit $b$ verbindet.
\end{satz}

%
% Homotopie
%
\subsection{Homotopie
\label{buch:integration:wegintegral:subsection:homotpie}}
Im vorangegangenen Abschnitt wurde klar, dass das unter geeigneten
Voraussetzungen an das Gebiet das Wegintegral für verschiedene Wege 
zwischen $a$ und $b$ gleich ist.
Die Bedingung war etwas schwerfällig zu formulieren, ergab sich aber
unmittelbar aus dem Satz von Green.
Es musste verlangt werden, zwei Wege zwischen $a$ und $b$ ein in
$U$ enthaltenes Teilgebiet beranden.

Eine etwas intuitivere Formulierung wäre, dass sich die beiden
Wege ineinander deformieren lassen.
Während der Deformation wird das Teilgebiet überstrichen.
Wir versuchen jetzt, dieses Idee mit einer mathematisch präzisen
Definition zu erfassen.
%
% fig-homotopie.tex
%
% (c) 2026 Prof Dr Andreas Müller
%
\begin{figure}
\centering
\includegraphics{chapters/030-integration/images/homotopie.pdf}
\caption{Eine Homotopie zweier Wege $\gamma_0$ und $\gamma_1$ in $U$
ist eine Abbildung $H\colon I\times I\to U$ derart, dass
$\gamma_i(t)=H(t,i)$ für $i\in\{0,1\}$ und festen Endpunkten.
\label{buch:integration:fig:homotopie}}
\end{figure}


\begin{definition}[Homotopie von Wegen]
\label{buch:integration:definition:homotpie}
Eine \emph{Homotopie} von Wegen $\gamma_0\colon I\to U$
\index{Homotpie}%
und
$\gamma_1\colon I\to U$, $I=[0,1]$,
in einem Gebiet $U$
mit gleichen Endpunkten
$\gamma_0(0)=\gamma_1(0)$
und
$\gamma_0(1)=\gamma_1(1)$
ist eine Abbildung
\[
H
\colon
I\times I \to U
:
(s,t) \mapsto H(s,t)
\]
mit $H(0,t)=\gamma_0(t)$ und $H(1,t)=\gamma_1(t)$
und konstanten Endpunkten
$H(s,0)=\gamma_i(0)$
und
$H(s,1)=\gamma_i(1)$.
Zwei Wege $\gamma_i\colon I\to U$  heissen \emph{homotop}, in Zeichen
\index{homotop}
$\gamma_0\sim \gamma_1$, wenn es eine
Homotopie gibt, die die beiden Wege verbindet.
\end{definition}

\begin{satz}
\label{buch:integration:satz:homotopie}
Sei $f\colon U\to\mathbb{C}$ eine holomorphe Funktion und $H$ eine Homotopie
der Wege $\gamma_i$, $i=0,1$, mit $H(s,0) = a$ und $H(s,1)=b$ für $s\in I$.
Dann ist
\[
\int_{\gamma_0} f(z)\,dz
=
\int_{\gamma_1} f(z)\,dz.
\]
\end{satz}

\begin{proof}
TODO
\end{proof}


